\documentclass{article}

\usepackage[spanish]{babel} 
\usepackage[numbers,sort&compress]{natbib}
\usepackage{graphicx}

\title{Reporte 1}
\author{Anthony Haziel Pérez Bernal}

\begin{document}
\maketitle{Tipos de curvas}
\section{Linea recta}
Llamamos linea recta al lugar geométrico de los puntos tales que tomados dos puntos diferentes cualesquiera del lugar, el valor de la pendiente m resulta siempre constante.

Geométricamente , una recta queda perfectamente
determinada por uno de sus puntos y su dirección. Analiticamente, la ecuación de una recta puede estar perfectamente determinada si se
conocen las coordenadas de uno de sus puntos y su angulo de inclinación (y , por lo tanto , su pendiente).

\section{Parabola}
La ecuación de la parabola se define como el lugar geométrico de un punto que se mueve de acuerdo con una ley especificada.

Una parabola es el lugar geométrico de un punto que se mueve en un plano de tal manera que su distancia de una recta fija, situada en el plano , es siempre igual a su distancia de un punto fijo del plano y que no pertenece a la recta.
El punto fijo se llama foco y la recta fija directriz de la parabola.

\section{Circunferencia} 
La circunferencia es el lugar geométrico de un punto que se mueve en un plano de tal manera que se conserva siempre a una distancia constante de un punto fijo de ese plano. El punto fijo se llama centro de la circunferencia, y la distancia
constante se llama radio.

\section{Elipse}
Una elipse es el lugar geométrico de un punto que se mueve en un plano de tal manera que la suma de sus distancias a dos puntos fijos de ese plano es siempre igual a una constante, mayor que la distancia entre 1os dos puntos. Los dos puntos fijos se llaman focos de la elipse. La definición de
una elipse excluye el caso en que el punto movil este sobre el segmento que une 1os focos.

\section{Hiperbola}
Una hipirbola es el lugar geometrico de un punto que se mueve en un plano de tal manera que el valor absoluto de la diferencia de sus distancias a dos puntos fijos del plano , llamados focos, es siempre igual a una cantidad constante, positiva y menor que la distancia entre 1os focos. 

\bibliography{biblio.bib}
\bibliographystyle{plainnat}

\end{document}
