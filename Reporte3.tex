\documentclass{article}

\usepackage[spanish]{babel} 
\usepackage[numbers,sort&compress]{natbib}
\usepackage{graphicx}

\title{Reporte 3}
\author{Anthony Haziel Pérez Bernal}

\begin{document}
\maketitle{Método de Bisección}
\section{Resumen}
Este es uno de los métodos más sencillos y de fácil
intuición, para resolver ecuaciones en una variable. Se basa en el Teorema de los Valores Intermedios, el cual establece que toda función continua f en un intervalo cerrado [a;b] toma todos los valores que
se hallan entre f (a) y f (b).

En caso de que f (a) y f (b) tengan signos opuestos (es decir, f (a)· f (b) < 0), el valor cero sería un valor intermedio entre f (a) y f (b).

El método consiste en lo siguiente: Supongamos que en el intervalo [a,b] hay un cero de f . Calculamos el punto medio m = (a +b)/2 del intervalo [a,b]. A continuación calculamos f (m). En caso de que f (m) sea igual a cero, ya hemos encontrado la solución buscada. En caso de que no lo sea, verificamos si f (m) tiene signo opuesto al de f (a). Se redefine el intervalo [a,b] como [a,m] o [m,b] según se haya determinado en cuál de estos intervalos ocurre un cambio de signo. A este nuevo intervalo se le aplica el mismo procedimiento y así, sucesivamente, iremos
encerrando la solución en un intervalo cada vez más pequeño, hasta alcanzar la precisión deseada.

\section{Ejemplos}

Tenemos primero para x**3, el rango que use para que hiciera la iteracion fue de [-1;1]. Figura 1.

\begin{figure}
\centering
    \includegraphics[width=1.25\textwidth]{1.png}
    \caption{Ejemplo 1}
    \label{fig_label}
\end{figure}

Despues para x**5-100*x**4+3995*x**3-79700*x**2+794004*x-3160075, el rango que use para la iteracion fue de [17;22.2]. Figura 2.

\begin{figure}
\centering
    \includegraphics[width=1.25\textwidth]{2.png}
    \caption{Ejemplo 2}
    \label{fig_label}
\end{figure}

Al final para x**3-2*x**2-5, ya encontrando sus raices use el rango de [-20;20] para mejor vista de la grafica. Figura 3.

\begin{figure}
\centering
    \includegraphics[width=1.25\textwidth]{3.png}
    \caption{Ejemplo 3}
    \label{fig_label}
\end{figure}

\section{bibliografia}
author = Anthony Haziel,
  title = Repositorio de Github,
  howpublished = "https://github.com/Anthony1606/matematicascomputacionales",
  year = 2021

\end{document}
